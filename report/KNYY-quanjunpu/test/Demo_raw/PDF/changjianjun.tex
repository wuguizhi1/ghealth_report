\input format.tex

\usepackage{graphicx}
\graphicspath{{cores/}}

\usepackage{environ}
\usepackage{colortbl,array,booktabs}
\usepackage{tabularx}

\colorlet{TablaBordeSuperior}{topcolor}
\colorlet{TablaBordeInferior}{topcolor}
\colorlet{TablaCentroSuperior}{blue!1}
\colorlet{TablaCentroInferior}{blue!20}
\colorlet{FuenteCabeceraTabla}{white}

\newcolumntype{M}[1]{>{\centering\arraybackslash}m{#1}}
%\newcommand{\tabularxcolumn}[1]{>\arraybackslash}m{#1}}

\tcbset{rtab/.style={
freelance,
frame code={
 \path[top color=topcolor,bottom color=topcolor]
   ([yshift=-#1*(\baselineskip+2pt)]interior.north west) --
   ([yshift=-#1*(\baselineskip+2pt)]interior.north east) {[sharp corners]--
    ([yshift=3pt]interior.north east) --
    ([yshift=3pt]interior.north west)} -- cycle;

  },
interior code={},
 }
}

\newcommand\fuentecabecera[1]{\textcolor{black}{\textbf{#1}}}

\begin{document}
 
\vspace*{4mm}
%% 各章节
\setlength{\arrayrulewidth}{.2pt}
\fontsize{8.8pt}{11pt}\selectfont
\color{black100}

\vspace*{3mm}

\noindent{\fontsize{15pt}{11pt}\selectfont{ 六、其他常见共生菌检测结果}}

%\vspace*{-4.25mm}
\fontsize{9pt}{11pt}\selectfont
\arrayrulecolor{gray2}
\begin{longtable}{|m{5.5cm}<{\centering}|m{4cm}<{\centering}|m{5cm}<{\centering}|}
\hline
\parbox[c]{\hsize}{\vskip8pt\centering\bf {检测内容} \vskip8pt} &  \bf{检测结果} &  \bf{参考范围} \\
\hline
\parbox[c]{\hsize}{\vskip8pt\centering {拟杆菌属 ~ (Bacteroides)} \vskip8pt} &
24.49 &
≤ 503.74 \\
\hline
\parbox[c]{\hsize}{\vskip8pt\centering {布劳特氏菌属 ~ (Blautia)} \vskip8pt} &
0.11 &
≥ 0.01 \\
\hline
\parbox[c]{\hsize}{\vskip8pt\centering {粪球菌属 ~ (Coprococcus)} \vskip8pt} &
8.98 &
≥ 0.01 \\
\hline
\parbox[c]{\hsize}{\vskip8pt\centering {瘤胃球菌属 ~ (Ruminococcus)} \vskip8pt} &
2.57 &
≥ 0.01 \\
\hline
\parbox[c]{\hsize}{\vskip8pt\centering {颤螺菌属 ~ (Oscillospira)} \vskip8pt} &
14.56 &
≥ 0.01 \\
\hline
\parbox[c]{\hsize}{\vskip8pt\centering {副拟杆菌属 ~ (Parabacteroides)} \vskip8pt} &
0.33 &
≥ 0.01 \\
\hline
\parbox[c]{\hsize}{\vskip8pt\centering {毛螺菌属 ~ (Lachnospira)} \vskip8pt} &
34.64 &
≥ 0.01 \\
\hline
\parbox[c]{\hsize}{\vskip8pt\centering {链球菌属 ~ (Streptococcus)} \vskip8pt} &
0.00 &
≤ 22.96 \\
\hline
\parbox[c]{\hsize}{\vskip8pt\centering {普雷沃氏菌属 ~ (Prevotella)} \vskip8pt} &
422.77 &
≤ 197.70 \\
\hline
\parbox[c]{\hsize}{\vskip8pt\centering {萨特氏菌属 ~ (Sutterella)} \vskip8pt} &
21.98 &
≤ 34.09 \\
\hline
\parbox[c]{\hsize}{\vskip8pt\centering {梭菌属 ~ (Clostridium)} \vskip8pt} &
1.12 &
≤ 4.54 \\
\hline
\parbox[c]{\hsize}{\vskip8pt\centering {假单胞菌属 ~ (Pseudomonas)} \vskip8pt} &
0.00 &
≤ 0.10 \\
\hline
\parbox[c]{\hsize}{\vskip8pt\centering {埃希氏菌属 ~ (Escherichia)} \vskip8pt} &
0.17 &
≤ 83.21 \\
\hline
\end{longtable}


\begin{spacing}{1.25}
\noindent\fontsize{9pt}{11pt}\selectfont {\ding{101} 本检测的结果仅供参考,无法代替医学诊断,实际情况请咨询拥有相应资质的临床医生;} \\
\noindent\fontsize{9pt}{11pt}\selectfont {\ding{101} 本项检测是基于当前微生物学的研究成果和国际公认的检测方法,但由于科学研究及检测方法的不断更新,本检测仍存在一定的局限性;} \\
\noindent\fontsize{9pt}{11pt}\selectfont {\ding{101} 检测结果仅对本次所送样负责。} \\

\end{spacing}

\end{document}
