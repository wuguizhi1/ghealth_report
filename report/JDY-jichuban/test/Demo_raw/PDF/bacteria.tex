\input format.tex

\usepackage{graphicx}
\graphicspath{{cores/}}

\usepackage{environ}
\usepackage{colortbl,array,booktabs}
\usepackage{tabularx}

\colorlet{TablaBordeSuperior}{topcolor}
\colorlet{TablaBordeInferior}{topcolor}
\colorlet{TablaCentroSuperior}{blue!1}
\colorlet{TablaCentroInferior}{blue!20}
\colorlet{FuenteCabeceraTabla}{white}

\newcolumntype{M}[1]{>{\centering\arraybackslash}m{#1}}
%\newcommand{\tabularxcolumn}[1]{>\arraybackslash}m{#1}}

\tcbset{rtab/.style={
freelance,
frame code={
 \path[top color=topcolor,bottom color=topcolor]
   ([yshift=-#1*(\baselineskip+2pt)]interior.north west) --
   ([yshift=-#1*(\baselineskip+2pt)]interior.north east) {[sharp corners]--
    ([yshift=3pt]interior.north east) --
    ([yshift=3pt]interior.north west)} -- cycle;

  },
interior code={},
 }
}

\newcommand\fuentecabecera[1]{\textcolor{black}{\textbf{#1}}}

\begin{document}

\vspace*{4mm}
%% 各章节
\setlength{\arrayrulewidth}{.2pt}
\fontsize{8.8pt}{11pt}\selectfont
\color{gray2}

\vspace*{3mm}

\noindent{\fontsize{9pt}{11pt}\selectfont\bf{ 三、主要有益菌检测结果}}

%\vspace*{-4.25mm}
\fontsize{8pt}{11pt}\selectfont
\arrayrulecolor{gray2}
\begin{longtable}{|m{2.8cm}<{\centering}|m{2cm}<{\centering}|m{2cm}<{\centering}|m{2cm}<{\centering}|m{4.9cm}<{\centering}|}
\hline
\begin{minipage}{2.7cm}\begin{center}{\vspace*{2mm} {\lantxh\bf 检测内容} \vspace*{2mm}}\end{center} \end{minipage} &
\begin{minipage}{2cm}\begin{center}{\lantxh\bf 检测结果}\end{center} \end{minipage} &
\begin{minipage}{2cm}\begin{center}{\lantxh\bf 检测值}\end{center} \end{minipage} &
\begin{minipage}{2cm}\begin{center}{\lantxh\bf 参考范围}\end{center} \end{minipage} &
\begin{minipage}{4.8cm}\begin{center}{\lantxh\bf 功能}\end{center} \end{minipage} \\
\hline
\begin{minipage}{2.7cm}\begin{center}{\vspace*{2mm} \lantxh 罗斯拜瑞氏菌属 \\
 (Roseburia) \vspace*{2mm}}
\end{center} \end{minipage} &
\begin{minipage}{2cm}\begin{center}{\lantxh 正常}\end{center} \end{minipage} &
\begin{minipage}{2cm}\begin{center}{\lantxh 8.37}\end{center} \end{minipage} &
\begin{minipage}{2cm}\begin{center}{\lantxh ≥ 2.32}\end{center} \end{minipage} &
\begin{minipage}{4.8cm}\begin{center}{\lantxh 产生丁酸等有益物质,抑制肠道炎症,有利于肠道及人体健康}\end{center} \end{minipage} \\
\hline
\begin{minipage}{2.7cm}\begin{center}{\vspace*{2mm} \lantxh 柔嫩梭菌属 \\
 (Faecalibacterium) \vspace*{2mm}}
\end{center} \end{minipage} &
\begin{minipage}{2cm}\begin{center}{\lantxh 正常}\end{center} \end{minipage} &
\begin{minipage}{2cm}\begin{center}{\lantxh 166.68}\end{center} \end{minipage} &
\begin{minipage}{2cm}\begin{center}{\lantxh ≥ 10.94}\end{center} \end{minipage} &
\begin{minipage}{4.8cm}\begin{center}{\lantxh 发酵纤维素产生丁酸等有益物质,抑制肠道炎症,促进肠道健康}\end{center} \end{minipage} \\
\hline
\begin{minipage}{2.7cm}\begin{center}{\vspace*{2mm} \lantxh 双歧杆菌属 \\
 (Bifidobacterium) \vspace*{2mm}}
\end{center} \end{minipage} &
\begin{minipage}{2cm}\begin{center}{\lantxh 正常}\end{center} \end{minipage} &
\begin{minipage}{2cm}\begin{center}{\lantxh 0.06}\end{center} \end{minipage} &
\begin{minipage}{2cm}\begin{center}{\lantxh ≥ 0.01}\end{center} \end{minipage} &
\begin{minipage}{4.8cm}\begin{center}{\lantxh 有益菌,降解人体不能消化的多糖,产乳酸,调节免疫及肠道环境}\end{center} \end{minipage} \\
\hline
\begin{minipage}{2.7cm}\begin{center}{\vspace*{2mm} \lantxh 乳酸杆菌属 \\
 (Lactobacillus) \vspace*{2mm}}
\end{center} \end{minipage} &
\begin{minipage}{2cm}\begin{center}{\lantxh 有害}\end{center} \end{minipage} &
\begin{minipage}{2cm}\begin{center}{\lantxh 0.00}\end{center} \end{minipage} &
\begin{minipage}{2cm}\begin{center}{\lantxh ≥ 0.01}\end{center} \end{minipage} &
\begin{minipage}{4.8cm}\begin{center}{\lantxh 肠道益生菌,能够生成乳酸,抑制有害菌及炎症,调节肠道环境}\end{center} \end{minipage} \\
\hline
\begin{minipage}{2.7cm}\begin{center}{\vspace*{2mm} \lantxh 阿克曼氏菌属 \\
 (Akkermansia) \vspace*{2mm}}
\end{center} \end{minipage} &
\begin{minipage}{2cm}\begin{center}{\lantxh 有害}\end{center} \end{minipage} &
\begin{minipage}{2cm}\begin{center}{\lantxh 0.00}\end{center} \end{minipage} &
\begin{minipage}{2cm}\begin{center}{\lantxh ≥ 0.01}\end{center} \end{minipage} &
\begin{minipage}{4.8cm}\begin{center}{\lantxh 降解粘蛋白、调节免疫,有利于肠黏膜完整性,保持正常体重}\end{center} \end{minipage} \\
\hline
\end{longtable}

\vspace*{0mm}

\begin{spacing}{1.25}
\noindent\fontsize{9pt}{11pt}\selectfont {综合您的肠道主要有益菌检测结果,乳酸杆菌属、阿克曼氏菌属的含量异常,
不利于抑制肠道炎症、调节肠道环境、调节免疫等,
需引起注意。} \\

\end{spacing}

\vspace*{8mm}

\noindent{\fontsize{9pt}{11pt}\selectfont\bf{ 四、主要有害菌检测结果}}

%\vspace*{-4.25mm}
\fontsize{8pt}{11pt}\selectfont
\arrayrulecolor{gray2}
\begin{longtable}{|m{2.8cm}<{\centering}|m{2cm}<{\centering}|m{2cm}<{\centering}|m{2cm}<{\centering}|m{4.9cm}<{\centering}|}
\hline
\begin{minipage}{2.7cm}\begin{center}{\vspace*{2mm} {\lantxh\bf 检测内容} \vspace*{2mm}}\end{center} \end{minipage} &
\begin{minipage}{2cm}\begin{center}{\lantxh\bf 检测结果}\end{center} \end{minipage} &
\begin{minipage}{2cm}\begin{center}{\lantxh\bf 检测值}\end{center} \end{minipage} &
\begin{minipage}{2cm}\begin{center}{\lantxh\bf 参考范围}\end{center} \end{minipage} &
\begin{minipage}{4.8cm}\begin{center}{\lantxh\bf 功能}\end{center} \end{minipage} \\
\hline
\begin{minipage}{2.7cm}\begin{center}{\vspace*{2mm} \lantxh 脱硫弧菌属 \\
 (Desulfovibrio) \vspace*{2mm}}
\end{center} \end{minipage} &
\begin{minipage}{2cm}\begin{center}{\lantxh 正常}\end{center} \end{minipage} &
\begin{minipage}{2cm}\begin{center}{\lantxh 2.79}\end{center} \end{minipage} &
\begin{minipage}{2cm}\begin{center}{\lantxh ≤ 4.13}\end{center} \end{minipage} &
\begin{minipage}{4.8cm}\begin{center}{\lantxh 产生硫化氢,刺激肠道产生炎症反应,不利于肠道健康}\end{center} \end{minipage} \\
\hline
\begin{minipage}{2.7cm}\begin{center}{\vspace*{2mm} \lantxh 多尔氏菌属 \\
 (Dorea) \vspace*{2mm}}
\end{center} \end{minipage} &
\begin{minipage}{2cm}\begin{center}{\lantxh 正常}\end{center} \end{minipage} &
\begin{minipage}{2cm}\begin{center}{\lantxh 1.23}\end{center} \end{minipage} &
\begin{minipage}{2cm}\begin{center}{\lantxh ≤ 10.09}\end{center} \end{minipage} &
\begin{minipage}{4.8cm}\begin{center}{\lantxh 肠道的主要产气菌之一,与肠易激综合征等疾病相关}\end{center} \end{minipage} \\
\hline
\begin{minipage}{2.7cm}\begin{center}{\vspace*{2mm} \lantxh 弯曲杆菌属 \\
 (Campylobacter) \vspace*{2mm}}
\end{center} \end{minipage} &
\begin{minipage}{2cm}\begin{center}{\lantxh 正常}\end{center} \end{minipage} &
\begin{minipage}{2cm}\begin{center}{\lantxh 0.00}\end{center} \end{minipage} &
\begin{minipage}{2cm}\begin{center}{\lantxh ≤ 0.20}\end{center} \end{minipage} &
\begin{minipage}{4.8cm}\begin{center}{\lantxh 多数菌种为致病菌,可引起弯曲菌病,表现为严重腹泻或痢疾综合征}\end{center} \end{minipage} \\
\hline
\end{longtable}

\begin{spacing}{1.25}
\noindent\fontsize{9pt}{11pt}\selectfont {综合您的肠道主要有害菌检测结果,各有害菌含量均处于正常范围,有害物质产生少、
肠道炎症风险低,整体状况良好。肠道菌群含量维持动态平衡,请您保持良好饮食及生活习惯,有助于维持现有的菌群状态。} \\

\end{spacing}

\end{document}
