\input format.tex

\usepackage{graphicx}
\graphicspath{{cores/}}

\usepackage{environ}
\usepackage{colortbl,array,booktabs}
\usepackage{tabularx}

\colorlet{TablaBordeSuperior}{topcolor}
\colorlet{TablaBordeInferior}{topcolor}
\colorlet{TablaCentroSuperior}{blue!1}
\colorlet{TablaCentroInferior}{blue!20}
\colorlet{FuenteCabeceraTabla}{white}

\newcolumntype{M}[1]{>{\centering\arraybackslash}m{#1}}
%\newcommand{\tabularxcolumn}[1]{>\arraybackslash}m{#1}}

\tcbset{rtab/.style={
freelance,
frame code={
 \path[top color=topcolor,bottom color=topcolor]
   ([yshift=-#1*(\baselineskip+2pt)]interior.north west) --
   ([yshift=-#1*(\baselineskip+2pt)]interior.north east) {[sharp corners]--
    ([yshift=3pt]interior.north east) --
    ([yshift=3pt]interior.north west)} -- cycle;

  },
interior code={},
 }
}

\newcommand\fuentecabecera[1]{\textcolor{black}{\textbf{#1}}}

\begin{document}

\vspace*{4mm}
%% 各章节
\setlength{\arrayrulewidth}{.2pt}
\fontsize{8.8pt}{11pt}\selectfont
\color{gray2}

\vspace*{3mm}

\noindent{\fontsize{9pt}{11pt}\selectfont\bf{ 五、致病菌检测结果}}

%\vspace*{-4.25mm}
\fontsize{8pt}{11pt}\selectfont
\arrayrulecolor{gray2}
\begin{longtable}{|m{3.2cm}<{\centering}|m{1.8cm}<{\centering}|m{1.8cm}<{\centering}|m{2cm}<{\centering}|m{4.9cm}<{\centering}|}
\hline
\begin{minipage}{3.2cm}\begin{center}{\vspace*{2mm} {\lantxh\bf 检测内容} \vspace*{2mm}}\end{center} \end{minipage} &
\begin{minipage}{1.7cm}\begin{center}{\lantxh\bf 检测结果}\end{center} \end{minipage} &
\begin{minipage}{1.7cm}\begin{center}{\lantxh\bf 检测值}\end{center} \end{minipage} &
\begin{minipage}{2cm}\begin{center}{\lantxh\bf 参考范围}\end{center} \end{minipage} &
\begin{minipage}{4.8cm}\begin{center}{\lantxh\bf 功能}\end{center} \end{minipage} \\
\hline
\begin{minipage}{3.2cm}\begin{center}{\vspace*{2mm} \lantxh 脆弱拟杆菌 \\
 (Bacteroides fragilis) \vspace*{2mm}}
\end{center} \end{minipage} &
\begin{minipage}{1.7cm}\begin{center}{\lantxh 有害}\end{center} \end{minipage} &
\begin{minipage}{1.7cm}\begin{center}{\lantxh 0.11}\end{center} \end{minipage} &
\begin{minipage}{2cm}\begin{center}{\lantxh ≤ 55.89}\end{center} \end{minipage} &
\begin{minipage}{4.8cm}\begin{center}{\lantxh 可能导致菌血症、腹内感染、腹膜炎}\end{center} \end{minipage} \\
\hline
\begin{minipage}{3.2cm}\begin{center}{\vspace*{2mm} \lantxh 病原性大肠埃希氏菌 \\
 (Escherichia coli) \vspace*{2mm}}
\end{center} \end{minipage} &
\begin{minipage}{1.7cm}\begin{center}{\lantxh 有害}\end{center} \end{minipage} &
\begin{minipage}{1.7cm}\begin{center}{\lantxh 0.17}\end{center} \end{minipage} &
\begin{minipage}{2cm}\begin{center}{\lantxh ≤ 83.21}\end{center} \end{minipage} &
\begin{minipage}{4.8cm}\begin{center}{\lantxh 可能导致肠胃炎、尿路感染、新生儿脑膜炎、腹膜炎等}\end{center} \end{minipage} \\
\hline
\end{longtable}

\vspace*{0mm}

\begin{spacing}{1.25}
\noindent\fontsize{9pt}{11pt}\selectfont {您的肠道内本次检测出2种潜在致病菌,其中2
的含量异常,当身体免疫力较强时可能不会出现不适感。但请注意,当您的免疫力较弱时,初期可能引起腹泻、腹痛、感染等病症,长期可能导致上表中所述疾病。建
议您持续监测,必要时请到正规医院咨询专科医生,遵从医生的建议做进一步检查。} \\

\end{spacing}

\vspace*{8mm}

\noindent{\fontsize{9pt}{11pt}\selectfont\bf{ 六、其他常见菌检测结果}}

%\vspace*{-4.25mm}
\fontsize{8pt}{11pt}\selectfont
\arrayrulecolor{gray2}
\begin{longtable}{|m{5.5cm}<{\centering}|m{4cm}<{\centering}|m{5cm}<{\centering}|}
\hline
\begin{minipage}{5.5cm}\begin{center}{\vspace*{1mm} {\lantxh\bf 检测内容} \vspace*{1mm}}\end{center} \end{minipage} &
\begin{minipage}{4cm}\begin{center}{\lantxh\bf 检测结果}\end{center} \end{minipage} &
\begin{minipage}{5cm}\begin{center}{\lantxh\bf 检测值}\end{center} \end{minipage} \\
\hline
\begin{minipage}{6cm}\begin{center}{\vspace*{1mm} \lantxh 拟杆菌属 ~ (Bacteroides) \vspace*{1mm}}\end{center} \end{minipage} &
\begin{minipage}{4cm}\begin{center}{\lantxh 24.49}\end{center} \end{minipage} &
\begin{minipage}{5cm}\begin{center}{\lantxh ≤ 503.74}\end{center} \end{minipage} \\
\hline
\begin{minipage}{6cm}\begin{center}{\vspace*{1mm} \lantxh 布劳特氏菌属 ~ (Blautia) \vspace*{1mm}}\end{center} \end{minipage} &
\begin{minipage}{4cm}\begin{center}{\lantxh 0.11}\end{center} \end{minipage} &
\begin{minipage}{5cm}\begin{center}{\lantxh ≥ 0.01}\end{center} \end{minipage} \\
\hline
\begin{minipage}{6cm}\begin{center}{\vspace*{1mm} \lantxh 粪球菌属 ~ (Coprococcus) \vspace*{1mm}}\end{center} \end{minipage} &
\begin{minipage}{4cm}\begin{center}{\lantxh 8.98}\end{center} \end{minipage} &
\begin{minipage}{5cm}\begin{center}{\lantxh ≥ 0.01}\end{center} \end{minipage} \\
\hline
\begin{minipage}{6cm}\begin{center}{\vspace*{1mm} \lantxh 瘤胃球菌属 ~ (Ruminococcus) \vspace*{1mm}}\end{center} \end{minipage} &
\begin{minipage}{4cm}\begin{center}{\lantxh 2.57}\end{center} \end{minipage} &
\begin{minipage}{5cm}\begin{center}{\lantxh ≥ 0.01}\end{center} \end{minipage} \\
\hline
\begin{minipage}{6cm}\begin{center}{\vspace*{1mm} \lantxh 颤螺菌属 ~ (Oscillospira) \vspace*{1mm}}\end{center} \end{minipage} &
\begin{minipage}{4cm}\begin{center}{\lantxh 14.56}\end{center} \end{minipage} &
\begin{minipage}{5cm}\begin{center}{\lantxh ≥ 0.01}\end{center} \end{minipage} \\
\hline
\begin{minipage}{6cm}\begin{center}{\vspace*{1mm} \lantxh 副拟杆菌属 ~ (Parabacteroides) \vspace*{1mm}}\end{center} \end{minipage} &
\begin{minipage}{4cm}\begin{center}{\lantxh 0.33}\end{center} \end{minipage} &
\begin{minipage}{5cm}\begin{center}{\lantxh ≥ 0.01}\end{center} \end{minipage} \\
\hline
\begin{minipage}{6cm}\begin{center}{\vspace*{1mm} \lantxh 毛螺菌属 ~ (Lachnospira) \vspace*{1mm}}\end{center} \end{minipage} &
\begin{minipage}{4cm}\begin{center}{\lantxh 34.64}\end{center} \end{minipage} &
\begin{minipage}{5cm}\begin{center}{\lantxh ≥ 0.01}\end{center} \end{minipage} \\
\hline
\begin{minipage}{6cm}\begin{center}{\vspace*{1mm} \lantxh 链球菌属 ~ (Streptococcus) \vspace*{1mm}}\end{center} \end{minipage} &
\begin{minipage}{4cm}\begin{center}{\lantxh 0.00}\end{center} \end{minipage} &
\begin{minipage}{5cm}\begin{center}{\lantxh ≤ 22.96}\end{center} \end{minipage} \\
\hline
\begin{minipage}{6cm}\begin{center}{\vspace*{1mm} \lantxh 普雷沃氏菌属 ~ (Prevotella) \vspace*{1mm}}\end{center} \end{minipage} &
\begin{minipage}{4cm}\begin{center}{\lantxh 422.77}\end{center} \end{minipage} &
\begin{minipage}{5cm}\begin{center}{\lantxh ≤ 197.70}\end{center} \end{minipage} \\
\hline
\begin{minipage}{6cm}\begin{center}{\vspace*{1mm} \lantxh 萨特氏菌属 ~ (Sutterella) \vspace*{1mm}}\end{center} \end{minipage} &
\begin{minipage}{4cm}\begin{center}{\lantxh 21.98}\end{center} \end{minipage} &
\begin{minipage}{5cm}\begin{center}{\lantxh ≤ 34.09}\end{center} \end{minipage} \\
\hline
\begin{minipage}{6cm}\begin{center}{\vspace*{1mm} \lantxh 梭菌属 ~ (Clostridium) \vspace*{1mm}}\end{center} \end{minipage} &
\begin{minipage}{4cm}\begin{center}{\lantxh 1.12}\end{center} \end{minipage} &
\begin{minipage}{5cm}\begin{center}{\lantxh ≤ 4.54}\end{center} \end{minipage} \\
\hline
\begin{minipage}{6cm}\begin{center}{\vspace*{1mm} \lantxh 假单胞菌属 ~ (Pseudomonas) \vspace*{1mm}}\end{center} \end{minipage} &
\begin{minipage}{4cm}\begin{center}{\lantxh 0.00}\end{center} \end{minipage} &
\begin{minipage}{5cm}\begin{center}{\lantxh ≤ 0.10}\end{center} \end{minipage} \\
\hline
\begin{minipage}{6cm}\begin{center}{\vspace*{1mm} \lantxh 埃希氏菌属 ~ (Escherichia) \vspace*{1mm}}\end{center} \end{minipage} &
\begin{minipage}{4cm}\begin{center}{\lantxh 0.17}\end{center} \end{minipage} &
\begin{minipage}{5cm}\begin{center}{\lantxh ≤ 83.21}\end{center} \end{minipage} \\
\hline
\end{longtable}

\begin{spacing}{1.25}
\noindent\fontsize{8pt}{11pt}\selectfont {\ding{101} 本报告为科研报告,内容仅供参考,不作为临床诊断使用,无法代替医学诊断,也无法作为用药参考,实际情况请咨询拥有相应资质的临床医生或执业药师;} \\
\noindent\fontsize{8pt}{11pt}\selectfont {\ding{101} 本项检测是基于当前微生物学的研究成果和国际公认的检测方法,但由于科学研究及检测方法的不断更新,本检测仍存在一定的局限性;} \\
\noindent\fontsize{8pt}{11pt}\selectfont {\ding{101} 检测结果仅对本次所送样负责。} \\

\end{spacing}

\end{document}
