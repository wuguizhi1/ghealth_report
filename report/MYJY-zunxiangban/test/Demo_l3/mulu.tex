\input format.tex

\setmainfont{Microsoft YaHei}   % 设置中文/粗体/斜体/主要字体

\begin{document}
\thispagestyle{empty}

\color{gray2}

\vspace*{0mm}

\heiti
{
\begin{center}
%{\fontsize{16.3pt}{17pt}\selectfont 报~告~目~录}
{\bf\sanhao 报~告~目~录}
\end{center}

\bigskip

\fontsize{9.3pt}{17pt}\selectfont

\tabcolsep=2pt
\begin{longtable}{m{1.4cm}m{14cm}}
第一部分 & {前言 \dotfill }\\
第二部分 & {检测结果 \dotfill 1}\\
 & \hspace*{2em}{肠道菌群 \dotfill }\\
 & \hspace*{4em}{分布 \dotfill }\\
 & \hspace*{4em}{致病菌 \dotfill }\\
 & \hspace*{2em}{消化和吸收 \dotfill }\\
 & \hspace*{4em}{消化和吸收总评 \dotfill }\\
 & \hspace*{4em}{物质代谢 \dotfill }\\
 & \hspace*{4em}{菌群代谢 \dotfill }\\
 & \hspace*{2em}{炎症和免疫 \dotfill }\\
 & \hspace*{2em}{相关疾病风险分析 \dotfill }\\
 & \hspace*{2em}{粪便状态 \dotfill }\\
第三部分 & {健康建议 \dotfill 2}\\
 & \hspace*{2em}{膳食方案 \dotfill 3}\\
 & \hspace*{2em}{肠道调节方案 \dotfill 4}\\
 & \hspace*{2em}{运动方案 \dotfill 4}\\
第四部分 & {附录 \dotfill }\\
 & \hspace*{2em}{\RNum{1}观大便~识健康 \dotfill }\\
 & \hspace*{2em}{\RNum{2}肠道菌群知多少 \dotfill }\\
 & \hspace*{2em}{\RNum{3}肠道菌群与健康风险 \dotfill }\\
 & \hspace*{2em}{\RNum{4}肠道菌群与肠道调养 \dotfill }\\
 & \hspace*{2em}{\RNum{5}膳食指南 \dotfill }\\
 & \hspace*{2em}{\RNum{6}参考列表 \dotfill }\\
\end{longtable}

}

\end{document}
